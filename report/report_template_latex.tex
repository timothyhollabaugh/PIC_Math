% THIS PART SET THE DESIGN OF THE REPORT. DO NOT MODIFY THIS SECTION: 

\documentclass[12pt]{article}
%%%%%%%%%%%%%%%%%%%%%%%%%%%%%%%%%%%%%%%%%%%%%%%%%%%%%%%%%%%%%%%%%%%%%%%%%%%%%%%%%%%%%%%%

\usepackage{amsmath, amssymb, latexsym, amscd, amsthm,amsfonts,amstext}
\usepackage[mathscr]{eucal}
\usepackage{graphicx}
\usepackage{subfig}
\usepackage{showlabels}
%\usepackage[pagewise]{lineno}\linenumbers
%\usepackage[nomarkers,figuresonly]{endfloat}


% set paper size:
 \textwidth = 16cm
 \textheight = 22cm
 \topmargin = -0.8cm
 \headsep =20pt
 \oddsidemargin = 15pt
 \evensidemargin = -15pt
 \renewcommand{\baselinestretch}{1.4} 
 
 
% define new mathematics environments: 
\newtheorem{theorem}{Theorem}[section]
\newtheorem{lemma}[theorem]{Lemma}
\newtheorem{definition}[theorem]{Definition}
\newtheorem{algorithm}[theorem]{Algorithm}
\newtheorem{remark}{Remark}[section]
\numberwithin{equation}{section}

\pagestyle{myheadings}
\makeatletter
\renewcommand\tableofcontents{    \@starttoc{toc}}
\makeatother


% Define some macros: 
\DeclareMathOperator{\Imag}{Im}
\DeclareMathOperator{\Real}{Re}

\def\bbC{\mathbb{C}}
\def\bbH{\mathbb{H}}
\def\bbR{\mathbb{R}}


% DOCUMENT STARTS: 
\begin{document}



% START TO MODIFY FROM HERE: 

\title{This is the title of your report}
\author{Your Name$^{1}$ and Another Name$^2$ \\
%EndAName
\\
$^1$Department of Mathematics, Rowan University, \\
201 Mullica Hill Rd, Glassboro, NJ 08028, USA. \\
Email: \texttt{Youremail@students.rowan.edu} \\
$^2$Department of Mathematics, Rowan University, \\
201 Mullica Hill Rd, Glassboro, NJ 08028, USA. \\
Email: \texttt{Youremail@students.rowan.edu} \\
}

\date{}
\maketitle

\begin{abstract}
One sentence abstract of your report. 
\end{abstract}

% -------------------------------------------------------
\textbf{Keywords}: Keywork 1, keywork 2. 



\section{Introduction} \label{sec:int} % this is called a label. We refer to each section via its label. 

You can refer to any section like this: section  \ref{sec:int}, section \ref{sec:2}, etc. 


When you discuss something from a reference (book, article, etc.) you have to cite it like this: see \cite{BKK}. 


The new article to be cited: \cite{Bento:JAND2017}

\section{Test}\label{sec:test}

\section{Problem statement}\label{sec:2}

A single equation can be written in the following form: 


\begin{equation}\label{eq3}
x + y = 3z.
\end{equation}


\begin{equation}\label{eq1}
x + y = z.
\end{equation}
To refer to that equation, use this: (\ref{eq1}). 



\begin{equation}\label{eq2}
x + y = 2z.
\end{equation}



\section{Results}\label{sec:result} 

A figure can be included in the text as follows. To refer to this figure, you can call, for example Figure \ref{fig1}.

\begin{figure}[tph]
\centering
\includegraphics[width=0.6\textwidth]{DLR_A380_1.jpg}
\caption{This is a caption of the figure.}
\label{fig1}
\end{figure}

See figure \ref{fig1}

A table can be created as follows: 

\begin{table}
\centering
\begin{tabular}{|c|c|c|} % three columns, text is placed in the center of each column.
\hline 
1 & 3 & 2\\
\hline 
2& 4 & 6 \\
\hline
\end{tabular}
\caption{Table 1: what is this table about?}
\label{table1}
\end{table} 
 
 

 \section*{Acknowledgment}
We will have to add an acknowledgment here. There is the phrase from PIC Math site to add to this section. 


% Using bibtex database: 
\bibliographystyle{abbrv}
\bibliography{references}


%\begin{thebibliography}{10}
%
%\bibitem{BKK}
%A.~Bakushinskii, M.~Klibanov, and N.~Koshev.
%\newblock Carleman weight functions for a globally convergent numerical method
%  for ill-posed cauchy problems for some quasilinear {PDEs}.
%\newblock {\em Nonlinear Analysis: Real World Applications}, 34:201--224, 2017.
%
%\end{thebibliography}
%



\end{document}
